\documentclass[12pt]{article}
\usepackage{amsmath,amssymb}
\usepackage[margin=2cm]{geometry}
\usepackage{float}
\usepackage{graphicx}
\usepackage[semicolon]{natbib}
\usepackage[multiple,bottom]{footmisc}
\usepackage[sc]{caption}
\usepackage{ctable}
\usepackage[para,online,flushleft]{threeparttable}

\linespread{1.1}

% Preferences for equation numbers
\numberwithin{equation}{section}

\title{Avoiding the Zero Lower Bound}
\author{Damian Romero}
\date{This version: March 24, 2017}

\begin{document}
	
\maketitle\thispagestyle{empty}

\begin{abstract}
%	\noindent In this paper I characterize the dynamics of an stylized version of the New Keynesian model when the economy is subject to demand shocks. First, I compare the evolution of the economy when the zero lower bound is an occasionally binding constraint with the case in which it is not taken into account, to state the main differences of both economies. I find that the constraint significantly reduces the capacity of the central bank to face the shock. Studying a modified Taylor rule, in which past inflation or past output are used to determine current policy, I find that the impact of the shock is reduced in at least 30\%, even when the weight of past macroeconomic variables in the Taylor rule is modest. This result is independent of the variable used by the central bank. The modified policy helps the economy to recover more quickly, avoiding extreme welfare losses. Finally, with the alternative rule, the nominal interest rate spends less than 0.5\% of the time on the zero lower bound, in comparison with a 2\% in the case of a traditional rule.

	\noindent In this paper, I evaluate an alternative to the standard Taylor rule typically used by central banks in a New Keynesian model subject to the zero lower bound. This modified Taylor rule includes past macroeconomic variables, such as output or inflation, so that the authority takes into account both past and current macroeconomic conditions in order to set the interest rate. I evaluate different variations of the rule, modifying how sensitive is the authority to past levels of inflation and output. The results suggest that including any of these past variables may generate substantial real and welfare gains with respect to the standard case. In particular, I find that the impact of the shock is reduced in at least 30\%, even when the weight of past macroeconomic variables in the Taylor rule is modest. This result is independent of the variable used by the central bank. When using the modified policy, the authority is able to reduce the probability of reaching extreme outcomes in the economy, like big and large recessions, which is translated in welfare gains both in the long-run and through the business cycle. Finally, with the alternative rule, the nominal interest rate spends less than 0.5\% of the time on the zero lower bound, in comparison with a 2\% in the case of a traditional rule.

\end{abstract}	
	
JEL classification: E30, E50, E60

\end{document}